
\chapter{基本アルゴリズム}
\label{chap:basic-algorithm}
\ref{sect:generalized-moore-graph}節で示した一般ムーアグラフの性質を利用して,
与えられた頂点数と次数の一般ムーアグラフを探索する基本的なアルゴリズムを
深さ優先探索をベースに開発する.後の章で探索空間を縮小する手法の比較対象となる.
本章では,一般的な深さ優先探索の説明をしたあと,これを一般ムーアグラフの探索に
適応する方法を示す.最後にアルゴリズムを示し,結果を与える.

\section{深さ優先探索}
\label{sect:depth-first-search}
深さ優先探索について説明する.理解している読者は\ref{sect:apply-to-gmg}
まで読み飛ばして差し支えない.

\section{一般ムーアグラフへの適応}
\label{sect:apply-to-gmg}

\subsection{初期グラフ}
\label{subsect:initial-graph}
探索の初期状態のグラフを与える.

\subsection{状態遷移}
\label{subsect:state-shifting}
定理\ref{theorem:gmg-geometric-property}を満たさないグラフを判定するため,
次の条件のどちらかを満足しない場合は,次の状態を作成しない.
\begin{enumerate}
\item フロンティアから出る頂点$v$に対して,$deg(v)\neq d$
\item 接続しようとしている辺$e=\{u,v\}$に対して,$dist(u,v)\leq 2Q$
\end{enumerate}

\subsection{アルゴリズム}
\label{subsect:basic-algorithm}
最後に,本節で説明したアルゴリズムをアルゴリズム
\ref{algo:basic-algorithm}に示す.
\begin{algorithm}
  \caption{基本アルゴリズム}
  \label{algo:basic-algorithm}
  \begin{algorithmic}
    \State{$a^2+b^2=c^2$}
  \end{algorithmic}
\end{algorithm}

\section{結果}
\label{sect:basic-result}
TBA
