
\chapter{基本アルゴリズム}
\label{chap:basic-algorithm}
\ref{sect:generalized-moore-graph}節で示した一般化ムーアグラフの性質を利用して,
与えられた頂点数と次数の一般化ムーアグラフを探索する基本的なアルゴリズムを
深さ優先探索をベースに開発する.後の章で探索空間を縮小する手法の比較対象となる.
本章では,一般的な深さ優先探索の説明をしたあと,これを一般化ムーアグラフの探索に
適応する方法を示す.最後にアルゴリズムを示し,結果を与える.

\section{深さ優先探索}
\label{sect:depth-first-search}
深さ優先探索について説明する.理解している読者は\ref{sect:apply-to-gmg}節
まで読み飛ばして差し支えない.

\section{一般化ムーアグラフへの適応}
\label{sect:apply-to-gmg}
本節では,\ref{sect:depth-first-search}で説明した深さ優先探索を用いて,
一般化ムーアグラフを発見する方法を述べる.

\subsection{初期グラフと辺の候補}
\label{subsect:initial-graph}
探索の初期状態のグラフを与える.以下,頂点集合を$V=\{1,\ldots,n\}$とする.
根からの距離が$i$である頂点の集合を,第$i$層と定義する.(準備でいいかも)
頂点$1$を根とし,葉を除く頂点の次数が$k$である木を構築する.
(これも定義でいいかも)

\begin{definition}
  いくつかの関数を与える.$\mathrm{TreeDepth}(v)$は,次数$k$の平衡木に
  おいて,頂点$v$がどの層に位置するのかを求める.その中身は,次式である.
  \[ \mathrm{TreeDepth}(v) = \begin{cases}
    0 & v = 1 \\
    \log_{k-1}{(\frac{(v-1)(k-2)}{k}+1)} + 1 & v > 1
  \end{cases} \]
  $\mathrm{NVertBTree}(l)$は,深さ$l$,次数$k$の
  平衡木の頂点数を計算する.その中身は,次式で与えられる.
  \[ \mathrm{NVertBTree}(l) = \begin{cases}
    1 & l = 0 \\
    k + 1 & l = 1 \\
    \frac{k(k-1)^l-1}{k-2} + 1 & l > 1
  \end{cases} \]
  $\mathrm{Parent}$関数は,次数$k$の平衡木において,頂点$v$の親の頂点番号を
  求める.次式で与えられる.
  \begin{align*}
    \mathrm{Parent}(v)=&\mathrm{NVertBTree}(\mathrm{TreeDepth}(v)-2) +\\
    &\frac{v-\mathrm{NVertBTree}(\mathrm{TreeDepth}(v)-1)}{k-1}
  \end{align*}
\end{definition}

必要な関数が定義できたので,次の操作で初期グラフ$T$を構築する.
$v=\{2,\ldots,\mathrm{NVertBTree}(Q)\}$に対して,$\mathrm{Parent}(v)$と
$v$を隣接させる.こうすることで,頂点数$n-R$,次数$k$の平衡木を構築できる.
このとき,残りの$R$個の頂点は孤立点であることに注意する.

\begin{example}
  頂点数12,次数3の初期グラフ$T$を考える.
  図\ref{fig:initial-tree-example}に頂点数12,次数3の初期グラフを示す.
  \begin{figure}
    \centering
    \def\svgwidth{.5\columnwidth}
    \input{initial-tree-example.pdf_tex}
    \caption{頂点数12,次数3の初期グラフ}
    \label{fig:initial-tree-example}
  \end{figure}
\end{example}

\subsection{可能な辺の追加/無追加}
\label{subsect:feasible-edge-(no)-addition}
初期グラフ$T$を構築した後,$T$上での次数が$k$未満の頂点同士を
接続させる辺の列を考える.そのような辺の列は次で与えられる.
\[ \bar{E} = \bar{e}_i = \{(v_i,w_i)\,|
\,d_T(v_i)<k,d_T(w_i)<k,v_i<w_i,v_i\in V,w_i\in V\} \]
探索では,$\bar{e}_1,\ldots,\bar{e}_m$と順番に辺を取り出し,
グラフに追加する/しないを判定して新たなグラフを作ることを繰り返す.

グラフ$G$と辺$\bar{e}_i$が与えられたとき,辺$\bar{e}_i$を追加した後のグラフ,
追加しない後のグラフが定理\ref{theorem:gmg-geometric-property}を満たすか
判定する方法を与える.そのため,いくつかの関数を導入する.
\begin{definition}
  頂点$v$に対して,それと接する辺$\bar{e}_i$の$i$の
  最小値$\mathrm{Enter(v)}$と最大値$\mathrm{Exit}(v)$を考える.
  $\mathrm{Enter}(v)$は探索中,初めて$v$と接する辺の番号を,
  $\mathrm{Exit}(v)$は探索中に最後に$v$と接する辺の番号をそれぞれ表す.
  $\mathrm{Enter}(v)$と$\mathrm{Exit}(v)$の具体的な式は,次で与えられる.
  \begin{align*}
    \mathrm{Enter}(v) &= \min(i\,|\,\{v,w\}=\bar{e}_i) \\
    \mathrm{Exit}(v) &= \max(i\,|\,\{v,w\}=\bar{e}_i)
  \end{align*}
\end{definition}

以上を踏まえて,辺$\bar{e}_i$を追加した後のグラフ,しない後のグラフが
定理\ref{theorem:gmg-geometric-property}を満たすか判定する条件を述べる.
辺を$\bar{e}_i=\{v,w\}$,追加前のグラフを$G$,
追加した(しない)後のグラフを$H$とする.
\begin{enumerate}
\item 次数条件
  \label{item:degree-constraint}
  \begin{enumerate}
  \item $d_H(v)\leq k$かつ$d_H(w)\leq k$
  \item $\mathrm{Exit}(x)=i$なる$x\in\bar{e}_i$について,$d_H(x)=k$
  \end{enumerate}
\item 閉路条件
  \label{item:cycle-constraint}
  \begin{enumerate}
  \item 追加するならば,$d_G(v,w)\geq2Q$\\
    この条件を満たすとき,$\bar{e}_i$によってできる閉路の最小の長さは$2Q+1$
    となり,定理\ref{theorem:gmg-geometric-property}を満たす.
  \end{enumerate}
\end{enumerate}
以上の条件をすべて満たしたときのみ,次の辺$\bar{e}_{i+1}$に対して同じ判定が
行われる.これを繰り返して,最後の辺$\bar{e}_{|\bar{E}|}$を追加するとき,
次の条件も加える.
\begin{enumerate}\setcounter{enumi}{2}
\item 直径条件
  \label{item:diameter-constraint}
  \begin{enumerate}
  \item $H$の直径は,$R>0$のとき$Q+1$,$R=0$のとき$Q$
  \end{enumerate}
\end{enumerate}

\begin{example}
  再び頂点数12,次数3の場合について考える.
  図\ref{fig:feasible-edges-example}に頂点数12,次数3の辺の候補を示す.
  \begin{figure}
    \centering
    \def\svgwidth{.5\columnwidth}
    \input{feasible-edges-example.pdf_tex}
    \caption{頂点数12,次数3の辺の候補}
    \label{fig:feasible-edges-example}
  \end{figure}
\end{example}

最後の辺に対する条件がすべて真のとき,アルゴリズムは一般化ムーアグラフを出力する.
判定するグラフがなくなったとき,アルゴリズムは一般化ムーアグラフが存在しない
ことを出力する.

\subsection{アルゴリズム}
\label{subsect:basic-algorithm}
最後に,本節で説明したアルゴリズムをアルゴリズム\ref{algo:basic-algorithm}
に示す.
\begin{algorithm}
  \caption{基本アルゴリズム}
  \label{algo:basic-algorithm}
  \begin{algorithmic}[1]
    \Require $n,k$
    \Ensure $M(n,k)$($\varnothing$if not found)
    \Procedure{Find generalized Moore Graph}{}
    \State $T\gets$\ref{subsect:initial-graph}で示した初期グラフ
    \State $\bar{E}\gets$\ref{subsect:initial-graph}で示した辺の候補
    \ForAll{$\mathrm{addflag}\in\{\top,\bot\}$}
    \State $I=$\Call{Search Part}{$T,1,\mathrm{addflag}$}
    \If{$I\neq\varnothing$}
    \State $\mathrm{return}\:I$
    \EndIf
    \EndFor
    \State $\mathrm{return}\:\varnothing$
    \EndProcedure
    \Procedure{Search Part}{$G,i,\mathrm{addflag}$}
    \If{$\mathrm{addflag}$}
    \State $H\gets G\cup\bar{e}_i$
    \Else
    \State $H\gets G$
    \EndIf
    \State $\mathrm{satisfy}\gets\bot$
    \If{$i<|\bar{E}|$かつ
      \ref{subsect:feasible-edge-(no)-addition}で示した
      条件\ref{item:degree-constraint},
      \ref{item:cycle-constraint}をすべて満たす
    }
    \State $\mathrm{satisfy}\gets\top$
    \EndIf
    \If{$i=|\bar{E}|$かつ
      \ref{subsect:feasible-edge-(no)-addition}で示した
      条件\ref{item:degree-constraint},
      \ref{item:cycle-constraint},
      \ref{item:diameter-constraint}をすべて満たす
    }
    \State $\mathrm{satisfy}\gets\top$
    \EndIf
    \If{$\mathrm{satisfy}$}
    \ForAll{$\mathrm{nextaddflag}\in\{\top,\bot\}$}
    \State $I=$\Call{Search Part}{$H,i+1,\mathrm{nextaddflag}$}
    \If{$I\neq\varnothing$}
    \State $\mathrm{return}\:I$
    \EndIf
    \EndFor
    \Else
    \State $\mathrm{return}\:\varnothing$
    \EndIf
    \EndProcedure
  \end{algorithmic}
\end{algorithm}

\section{実験}
\label{sect:exp-basic-algorithm}

\section{結果}
\label{sect:result-basic-algorithm}
TBA
