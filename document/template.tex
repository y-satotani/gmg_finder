\documentclass[11pt]{jarticle}
%
\usepackage{graphicx}
\usepackage{amsmath}
\usepackage{amssymb}
\usepackage{bm}
\usepackage{latexsym}
%
\addtolength{\textwidth}{40mm}
\addtolength{\oddsidemargin}{-20mm}
\addtolength{\evensidemargin}{-20mm}
\addtolength{\textheight}{15mm}
\addtolength{\topmargin}{-10mm}
%
\pagestyle{empty}
%
\title{研究進捗報告}
\author{高橋 研太郎}
%\date{\today}
\date{平成27年10月21日}
%
\begin{document}
%
\maketitle
\thispagestyle{empty}
%
\section{研究全体の目標}
データストリームにおいて一定期間あたりの出現回数が急激に増加または減少するデータを検出するストリームアルゴリズムを開発し,理論と実験の両面からその性能を評価する.さらに,Twitter等の実データに対して開発したアルゴリズムを適用し,有効性を検証する.
\section{前回打ち合わせ時に定めた短期目標}
\begin{enumerate}
\item 出現回数の倍率の推定値の誤差を計算するプログラムを作成する.
\item 使用可能なメモリ量が与えられたときのパラメータ自動設定プログラムを作成する.
\end{enumerate}
\section{本日までの進捗状況}
\begin{enumerate}
\item プログラムが完成した.また,すべてのデータセットに対し,ストリームアルゴリズムの倍数の推定値の誤差を計算し直した.結果はCSV形式で保存している.
\item パラメータの値を求めるための非線形方程式を解くプログラムを作成中であるが,期待通りの結果が得られていない.現在,その原因を調査中である.1週間以内には解決したい.
\end{enumerate}
\end{document}
