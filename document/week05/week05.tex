\documentclass[11pt]{jarticle}
%
\usepackage[dvipdfmx]{graphicx}
\usepackage{amsmath}
\usepackage{amssymb}
\usepackage{bm}
\usepackage{latexsym}
\usepackage{float}
\usepackage{subfig}
\usepackage[justification=centering]{caption}
%
\addtolength{\textwidth}{40mm}
\addtolength{\oddsidemargin}{-20mm}
\addtolength{\evensidemargin}{-20mm}
\addtolength{\textheight}{15mm}
\addtolength{\topmargin}{-10mm}
%
\pagestyle{empty}
%
\title{研究進捗報告}
\author{里谷 佳紀}
%\date{\today}
\date{平成29年10月17日}
%
\begin{document}
%
\maketitle
\thispagestyle{empty}
%
\section{研究全体の目標}
深さ優先探索をベースにした,一般ムーアグラフを発見するアルゴリズムを開発する.
さらに,初期グラフの改良や枝刈りを導入する,更なる改良案を提案する.
同時に,これらの改良を評価する.
\section{前回打ち合わせ時に定めた短期目標}
\begin{enumerate}
\item 展開ノード数の計数による評価プログラムの機能向上
\item 直径の下界を計算し,枝刈りを行うアルゴリズムの実装
\item 辺を削除することによる,頂点間距離の更新の方法の開発
\item 定理およびアルゴリズムの文書化
  \begin{enumerate}
  \item $2Q$以下の閉路が存在しないことの定理
  \item 基本アルゴリズム
  \item 辺を削除することによる,頂点間距離の更新の方法
  \end{enumerate}
\end{enumerate}
\section{本日までの進捗状況}
\begin{enumerate}
\item 展開ノード数の計数プログラムが完成した.それに伴い,再実験を実施した.
  結果を図\ref{fig:result}に示す.
\item 実装を完了し,実験を行った.結果は図\ref{fig:result}に示してある.
\item アルゴリズムを開発し,簡単なテストを行った.
  頂点数70の完全グラフから辺を逐次削除し,提案アルゴリズムの結果と真値とを
  比較し,間違いがないことを確認した.プログラムは別紙参照.
\item 完成したものがある.
  \begin{enumerate}
  \item 未着手
  \item 部分的に書いた.
  \item 大部分が仕上がった.別紙参照.
  \end{enumerate}
\end{enumerate}

\begin{figure}[H]
  \centering
  \subfloat[展開ノード数]{
    \includegraphics[width=.45\linewidth]{../../source/experiment/exp_cmp_algo_result_node.pdf}
  }\hfill
  \subfloat[平均実行時間(秒)]{
    \includegraphics[width=.45\linewidth]{../../source/experiment/exp_cmp_algo_result_time.pdf}
  }
  \caption{アルゴリズムの性能の比較 \\
    basicは基本アルゴリズムを,
    cycleは$2Q+2$の長さの閉路を追加したグラフを初期グラフとする方法を,
    minmaxは直径の下界を計算し$Q+1$より大きければ枝刈りする方法を,
    streeは全域木を初期グラフとする方法を表す.
  }
  \label{fig:result}
\end{figure}

\end{document}
