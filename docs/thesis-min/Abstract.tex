
近年の急速なWebアプリケーションの発展に伴いデータセンタの重要度が高まっている.
データセンタではスイッチの相互接続による計算機ネットワークが利用されており,
それは一般に無向正則グラフでモデル化される.また,
ネットワークの性能はグラフの接続構造と関連しており,
特にデータ通信の遅延は平均頂点間距離と密接な関係をもつ.
そのため,平均頂点間距離の小さい無向正則グラフの構築は
低遅延なネットワークの構築のための重要な課題である.

平均頂点間距離が理論的下界と一致する正則グラフは一般化ムーアグラフと呼ばれる.
一般化ムーアグラフが存在するか否かは頂点数と次数の組合せに依存することが
知られているが,存在するための頂点数と次数に関する条件は完全には解明されていない.
現在までに,指定された頂点数と次数をもつ一般化ムーアグラフを探索する方法が
提案されてきた.しかし,それらの方法には,一般化ムーアグラフの性質を十分に
利用していないという問題がある.

そこで,本研究は,頂点数と次数の任意の組に対応し,その性質を利用して
効率的に一般化ムーアグラフを探索する方法を提案し,その有効性を実験で検証する.
提案法は,全探索を基本として,初期グラフの制限や探索木の枝刈りによって
探索を効率的に行うものである.実験により,全域木から探索を開始することと,
直径の下界を用いた枝刈りが一般的に有効であることを示す.
