
近年の急速なWebアプリケーションの発展に伴い,データセンタや大規模グラフ分析が
注目されている.これらの技術はスイッチの相互接続による計算機ネットワークが必要
である.ネットワークの性能はトポロジ(接続構造)によって変化する.
トポロジについて議論する際,ネットワークを正則グラフとして表現することが多い.
低遅延なネットワークを実現するグラフは,平均頂点間距離の値が小さいグラフで,
そのために平均頂点間距離が小さなグラフの構築が重要である.

平均頂点間距離が理論的な下界と一致する正則グラフは一般化ムーアグラフと呼ばれる.
現在まで,そのようなグラフを構築する方法がいくつか提案されてきた.しかし,
すべての頂点数と次数の組に対応することと,一般化ムーアグラフの性質を利用して
効率的に構築することの両方を満たす構築方法はまだ提案されていない.

そこで,本研究は,任意の頂点数と次数の組に対応し,その性質を利用して
効率的に一般化ムーアグラフを構築する方法を提案し,その有効性を実験で検証する
ことを目的とする.方法を詳細に説明すると,全探索を基本として,初期状態の変更や
枝刈りによって探索を高効率に行うものである.

実験の結果,頂点数と次数がそれぞれ$16$と$4$程度の場合,
全域木から探索を開始することと,直径の下界を用いた枝刈りが有効であることが
確認できた.

