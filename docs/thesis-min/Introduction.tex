
\chapter{序論}
近年,Webアプリケーションの発展により,ユーザの活動によって生み出される膨大な
データの収集と活用が大いに注目されている.それを実現する技術の一つに
データセンタが挙げられる.データセンタにはサーバが多数存在し,それぞれの
サーバがスイッチと接続し,さらにスイッチが相互に接続してネットワークを形成している
\cite{Greenberg2009,Al-Fares2008}.
ネットワークの性能はトポロジ(スイッチの接続構造)によって
変化するため,適切なトポロジを選択する必要がある.
最近,データセンタのネットワークにランダムトポロジを利用することが提案されている
\cite{Singla2011,Koibuchi2012}.このトポロジは,Fat-Tree\cite{Al-Fares2008}
などの従来のトポロジと比べてスケーラビリティで優れているなどの特徴がある
\cite{Singla2011}.

低遅延性はデータセンタのスループットに直結する重要な性質である.
したがって,低遅延を実現するネットワークの構築が重要である.
スイッチを頂点とみなしてネットワークを無向正則グラフでモデル化すると,
遅延は平均頂点間距離と密接に関連する.現在まで,
平均頂点間距離が小さいグラフの構築に関する研究が行われてきた.
例えば,2015年に藤田らは,辺の入れ替えを繰り返しながら
グラフの平均頂点間距離を小さくする方法を開発した\cite{Fujita2015}.
しかし,この方法では局所解に陥ることが多く,平均頂点間距離が最小である
グラフを発見することが困難である.

一方で,1973年にCerfらは,平均頂点間距離が小さい正則グラフとして一般化ムーアグラフを
提案し\cite{Cerf1973},翌年に一般化ムーアグラフの平均頂点間距離は
同じ頂点数と次数の正則グラフの中で下界であることを証明した\cite{Cerf1974Lower}.
一般化ムーアグラフを求める方法はいくつか知られている.
例えば,
2016年に山本らは正則グラフを列挙する佐藤らの方法\cite{Sato2008}を応用し,
一般化ムーアグラフが発見されるまで列挙を続ける方法を開発した
\cite{Yamamoto2016}.
しかし,この方法は,一般化ムーアグラフがもつ性質を利用して効率的に探索できて
いるとは言えない.

本研究では,与えられた頂点数と次数をもつ一般化ムーアグラフの
効率的探索法を提案し,その有効性を検証する.
まず,一般化ムーアグラフの基本的性質を説明し,
それに続いて,深さ優先探索に基づく探索アルゴリズムを与える.
その後,探索の初期状態の制限と枝刈りの導入によって探索空間を削減する手法を提案し,
それによって探索が効率化されることを実験的に示す.
