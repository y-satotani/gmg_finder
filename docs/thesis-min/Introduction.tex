
\chapter{序論}
近年,Webアプリケーションの発展により,ユーザの活動によって生み出される膨大な
データの収集と活用が大いに注目されている.それを実現する技術の一つに,
データセンターが挙げられる.データセンターにはサーバが多数存在し,それぞれの
サーバがスイッチと接続し,さらにスイッチが相互に接続してネットワークを形成している
\cite{Greenberg2009,Al-Fares2008}.
ネットワークの性能はトポロジ(スイッチの接続構造)によって
変化する.最近,注目されるトポロジの一つにランダムトポロジがある
\cite{Singla2011,Koibuchi2012}.このトポロジは,Fat-Tree\cite{Al-Fares2008}
などの従来のトポロジと比べてスケーラビリティで優れているなどの特徴がある
\cite{Singla2011}.
その中でも,低遅延性はデータセンタのスループットに直結する重要な特性である.

このことから,低遅延なネットワークの構築が重要であることが分かる.
ネットワークを,スイッチを頂点とみなしたグラフで表現するとき,遅延は
平均頂点間距離と密接に関連する.現在まで,
平均頂点間距離が小さいグラフの構築に関する研究が行われてきた.
2015年,藤田らは,辺の入れ替えを繰り返しながら
グラフの平均頂点間距離を小さくする方法を開発した\cite{Fujita2015}.
ところが,この方法では局所解に陥ることが多く,平均頂点間距離が最小である
グラフを発見することが困難である.

一方で,1973年,Cerfらは,平均頂点間距離が小さい正則グラフとして一般化ムーアグラフを
提案し\cite{Cerf1973},翌年に一般化ムーアグラフの平均頂点間距離が
同じ頂点数と次数の正則グラフの中で下界であることを導出した\cite{Cerf1974Lower}.
一般化ムーアグラフを求める方法はいくつか知られている.
2004年,Sampelsは頂点推移グラフの中に一般化ムーアグラフが存在することを応用した
方法を開発した\cite{Sampels2004}.しかし,この方法では,任意の頂点数と次数の
一般化ムーアグラフを求めることはできない.
2016年,山本らは,正則グラフを列挙する佐藤らの方法\cite{Sato2008}を応用し,
途中で一般化ムーアグラフが発見されるまで列挙を繰り返す方法を開発した
\cite{Yamamoto2016}.
しかし,この方法は,一般化ムーアグラフがもつ性質を利用して効率的に探索できて
いないと言える.

そこで,本研究は,与えられた頂点数と次数をもつ一般化ムーアグラフの
効率的探索方法を提案し,その有効性を検証する.
まず,一般化ムーアグラフについて,その性質を説明した後,
深さ優先探索に基づく探索アルゴリズムを示す.
その後,探索空間の削減による効率化の手法を提案し,その効果を検証する.
具体的には,探索の初期状態の変更と枝刈りの導入である.

