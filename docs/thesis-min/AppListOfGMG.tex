
\chapter{一般化ムーアグラフの存在}
\label{chap:list-of-gmg}
全域木初期グラフと枝刈りを用いて,与えられた頂点数と次数をもつ一般化
ムーアグラフが存在するかを求めるプログラムを作成した.
頂点数と次数の組と一般化ムーアグラフの存在の関係を
図\ref{fig:gmg-existence}に示す.
図\ref{fig:gmg-existence}において,空白部分は正則グラフが存在しないことを
意味している.
ここで,予想\ref{conj:spanning-tree}の真偽によっては一般化ムーアグラフが
存在しないと示された組が無効になることに注意する.
一般化ムーアグラフの実体は次のURLに公開している.
\begin{verbatim}
github.com/y-satotani/gmg-finder/tree/master/docs/res/graph
\end{verbatim}

\begin{figure}[htbp]
  \centering
  \captionsetup{justification=centering}
  \includegraphics{gmg-existence-h.pdf}
  \caption{頂点数と次数の組と一般化ムーアグラフの存在の関係}
  \label{fig:gmg-existence}
\end{figure}
