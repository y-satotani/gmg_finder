
\chapter{結論}
本論文では,全探索法を基にした一般化ムーアグラフの探索の高速化および高効率化に
ついて述べた.まず,一般化ムーアグラフを定義して,その性質を示し証明を与えた.
次に,この性質を利用して,一般化ムーアグラフを探索するアルゴリズムを定めた.
最後に,探索空間の削減による高効率化の方法を与え,その効果を実験によって示した.

本論文では,高効率化の方法として,探索の初期状態を変更することと
枝刈りの二つを提案した.
探索の初期状態の変更とは,初期状態のグラフに予めいくつかの辺を追加した状態を
新たな初期状態とすることである.本論文では,辺を二本追加して閉路を含むようにした
グラフと,全域木の二種類を提案した.実験の結果から,一般的な場合では全域木から
探索を始める場合が最も効率が良いことが分かった.また,閉路を含むグラフは
ある頂点数と次数の組合せにおいて,効率が悪くなることが示された.
また,枝刈りの方法として,探索途中のグラフの直径の下界を用いる方法を提案した.
枝刈りによって探索空間が削減され高効率化は達成されたが,
頂点数および次数が小さい場合に探索に時間がかかることが結果から示された.

今後の課題は,提案した方法を用いて一般化ムーアグラフの存在を調べることや,
提案した初期状態から探索を開始することの妥当性の証明,
および,一般化ムーアグラフが存在しない時の,平均頂点間距離最小の正則グラフの
研究である.
