\chapter{gmg\_finder マニュアル}\label{chap-experimental-program}

一般化ムーアグラフを探索するためのライブラリ

\section*{依存ライブラリ・ソフトウェア}\label{ux4f9dux5b58ux30e9ux30a4ux30d6ux30e9ux30eaux30bdux30d5ux30c8ux30a6ux30a7ux30a2}

\begin{itemize}
\tightlist
\item
  \texttt{gcc}
\item
  \texttt{makefile}
\item
  \texttt{igraph}
\item
  \texttt{python} (一部)
\end{itemize}

\section*{実験プログラム}\label{ux5b9fux9a13ux30d7ux30edux30b0ux30e9ux30e0}

プログラムをビルドするには,\texttt{git}で
\texttt{https://github.com/y-satotani/gmg-finder}をクローンする.
すべての実験は\texttt{src/experiment}ディレクトリに含まれている.
\texttt{src}ディレクトリで\texttt{make\ ;\ make\ install}を実行すると,
以下の実行ファイルが\texttt{bin}ディレクトリに生成される.

\begin{itemize}
\tightlist
\item
  \texttt{exp\_cmp\_algo.out}
\item
  \texttt{exp\_cmp\_algo\_full.out}
\item
  \texttt{exp\_dist\_delete.out}
\item
  \texttt{exp\_miner.out}
\item
  \texttt{exp\_cmp\_algo\_enum.out}
\item
  \texttt{exp\_enumer.out}
\end{itemize}

以下,各プログラムを順番に説明する.

\subsection*{\texorpdfstring{\texttt{exp\_cmp\_algo.out}}{exp\_cmp\_algo.out}}\label{expux5fcmpux5falgo.out}

初期グラフと枝刈りの方法を変えて,性能を測定する.
実行例と結果例は以下のとおり.

\begin{verbatim}
src> experiment/exp_cmp_algo.out basic basic basic 10 3
\end{verbatim}

\begin{verbatim}
10,3,2,0,150,basic,basic,basic,150,15,16,0.000144
\end{verbatim}

結果は,次の値が\texttt{,}で区切られている.

\begin{itemize}
\tightlist
\item
  頂点数
\item
  次数
\item
  \texttt{Q} (\texttt{gmgf::graph\_config::Q()})
\item
  \texttt{R} (\texttt{gmgf::graph\_config::R()})
\item
  頂点間距離の総和の下界
\item
  グラフ初期化子
\item
  \texttt{basic}=\texttt{gmgf::basic\_graph\_initr}
\item
  \texttt{cycle}=\texttt{gmgf::cycle\_graph\_initr}
\item
  \texttt{stree}=\texttt{gmgf::stree\_graph\_initr}
\item
  候補辺ソート
\item
  \texttt{basic}=ソートなし
\item
  \texttt{sorted}=\texttt{gmgf::sorted\_graph\_initr}でソート
\item
  状態初期化子
\item
  \texttt{basic}=\texttt{gmgf::basic\_state\_initr}
\item
  \texttt{minmax}=\texttt{gmgf::minmax\_state\_initr}
\item
  \texttt{matrix}=\texttt{gmgf::mmmtr\_state\_initr}
\item
  結果のグラフの頂点間距離の総和
\item
  候補辺の数
\item
  展開した状態数
\item
  実行時間 (\texttt{gmgf::dfs\_finder::next()}の実行時間)
\end{itemize}

次のコマンドで,複数の結果を含むcsvファイルを生成できる.
\texttt{-\/-max-procs=x}の\texttt{x}の値だけ並列に実行する.

\begin{verbatim}
(echo n,d,Q,R,sspl_lb,ginitr,sorted,sinitr,sspl,edge,node,time ;\
 ./exp_cmp_algo_param.py |\
 xargs --max-lines=1 --max-procs=4 ./exp_cmp_algo.out) >\
cmp-algo.csv
\end{verbatim}

パラメータは,\texttt{exp\_cmp\_algo\_param.py}で生成される.
別のパラメータで実験する場合は,このファイルを編集する.

\subsection*{\texorpdfstring{\texttt{exp\_cmp\_algo\_full.out}}{exp\_cmp\_algo\_full.out}}\label{expux5fcmpux5falgoux5ffull.out}

\texttt{exp\_cmp\_algo.out}では探索開始から(もしあれば)最初の一個の
一般化ムーアグラフを得るまでの情報を計測したが,
\texttt{exp\_cmp\_algo\_full.out}では探索開始からすべての一般化ムーアグラフ
を得るまでの情報を計測する.

\begin{verbatim}
./exp_cmp_algo_full.out basic basic basic 10 3
\end{verbatim}

の実行例は次のとおり.

\begin{verbatim}
10,3,2,0,150,basic,basic,basic,4,150,150,15,75,0.00042
\end{verbatim}

結果は,次の値が\texttt{,}で区切られている.

\begin{itemize}
\tightlist
\item
  頂点数
\item
  次数
\item
  \texttt{Q} (\texttt{gmgf::graph\_config::Q()})
\item
  \texttt{R} (\texttt{gmgf::graph\_config::R()})
\item
  頂点間距離の総和の下界
\item
  グラフ初期化子 (\texttt{exp\_cmp\_algo.out}と同じ)
\item
  候補辺ソート (\texttt{exp\_cmp\_algo.out}と同じ)
\item
  状態初期化子 (\texttt{exp\_cmp\_algo.out}と同じ)
\item
  結果のグラフの頂点間距離の総和
\item
  得られたグラフの数 (同型グラフを重複して数える)
\item
  最大の頂点間距離
\item
  最小の頂点間距離
\item
  候補辺の数
\item
  展開した状態数
\item
  実行時間
\end{itemize}

\subsection*{\texorpdfstring{\texttt{exp\_dist\_delete.out}}{exp\_dist\_delete.out}}\label{expux5fdistux5fdelete.out}

一辺削除時の頂点間距離の更新の実験を行う.次の項目を入力とする.

\begin{itemize}
\tightlist
\item
  ネットワークトポロジ
\item
  random=Erd\H{o}s--R{\'e}nyiモデル
\item
  scale-free=Barab{\'a}si--Albertモデル
\item
  頂点数
\item
  ネットワークパラメータ
\item
  シード値 (ランダム関数の)
\end{itemize}

\begin{verbatim}
./exp_dist_delete.out random 1000 0.3 0
\end{verbatim}

の実行例は以下のとおり.

\begin{verbatim}
1000,0.3,random,0,150422,1,0.827778,0.076536
\end{verbatim}

結果は次の値が\texttt{,}で区切られている.

\begin{itemize}
\tightlist
\item
  頂点数
\item
  ネットワークパラメータ (トポロジにより,pかm)
\item
  ネットワークトポロジ
\item
  \texttt{random}=Erd\H{o}s--R{\'e}nyiモデル
\item
  \texttt{scale-free}=Barab{\'a}si--Albertモデル
\item
  シード値
\item
  辺の数
\item
  結果が一致したかどうか (1なら一致)
\item
  \texttt{igraph}ライブラリでの計算時間
\item
  提案手法の計算時間
\end{itemize}

\subsection*{\texorpdfstring{\texttt{exp\_miner.out}}{exp\_miner.out}}\label{expux5fminer.out}

\texttt{exp\_cmp\_algo.out}で,グラフ初期化子を\texttt{gmgf::stree\_graph\_initr},
状態初期化子を\texttt{gmgf::minmax\_state\_initr}とし,
さらに(もしあれば)一般化ムーアグラフを出力する機能を付加したプログラム.

\begin{verbatim}
./exp_miner.out 10 3
\end{verbatim}

の実行例は以下のとおり,

\begin{verbatim}
10,3,2,0,150,150,15,16,0.000452
\end{verbatim}

結果は次の値が\texttt{,}で区切られている.

\begin{itemize}
\tightlist
\item
  頂点数
\item
  次数
\item
  \texttt{Q} (\texttt{gmgf::graph\_config::Q()})
\item
  \texttt{R} (\texttt{gmgf::graph\_config::R()})
\item
  頂点間距離の総和の下界
\item
  結果のグラフの頂点間距離の総和
\item
  候補辺の数
\item
  展開した状態数
\item
  実行時間 (\texttt{gmgf::dfs\_finder::next()}の実行時間)
\end{itemize}

さらに,\texttt{n\textless{}頂点数\textgreater{}-d\textless{}次数\textgreater{}-miner.elist}の形式で,一般化ムーアグラフの
辺リストを保存する.

\subsubsection*{\texorpdfstring{\texttt{exp\_miner.py}}{exp\_miner.py}}\label{expux5fminer.py}

\texttt{exp\_miner.out}に並列計算とタイムアウト機能を追加したプログラム.
コマンドライン引数は以下のとおり.

\begin{itemize}
\tightlist
\item
  \texttt{-n}=頂点数((次数+1)\ldots{}Nの範囲を対象とする)
\item
  \texttt{-d}=次数(3\ldots{}Dの範囲を対象とする)
\item
  \texttt{-i}=結果ファイルリスト(これらの結果に含まれている頂点数次数組は計算しない)
\item
  \texttt{-o}=結果ファイル(指定しなければ標準出力)
\item
  \texttt{-t}=タイムアウト時間(単位:秒)
\end{itemize}

結果は,\texttt{exp\_miner.out}の結果を列としたcsvファイルである.

\subsection*{\texorpdfstring{\texttt{exp\_cmp\_algo\_enum.out}}{exp\_cmp\_algo\_enum.out}}\label{expux5fcmpux5falgoux5fenum.out}

初期グラフと枝刈りの方法を変えて,互いに非同型な一般化ムーアグラフを列挙する.
実行例と結果例は以下のとおり.

\begin{verbatim}
src> experiment/exp_cmp_algo_enum.out basic basic basic 10 3
\end{verbatim}

\begin{verbatim}
10,3,2,0,150,basic,basic,basic,1,1,15,23,0.030401
\end{verbatim}

結果は,次の値が\texttt{,}で区切られている.

\begin{itemize}
\tightlist
\item
  頂点数
\item
  次数
\item
  \texttt{Q} (\texttt{gmgf::graph\_config::Q()})
\item
  \texttt{R} (\texttt{gmgf::graph\_config::R()})
\item
  頂点間距離の総和の下界
\item
  グラフ初期化子
\item
  \texttt{basic}=\texttt{gmgf::basic\_graph\_initr}
\item
  \texttt{cycle}=\texttt{gmgf::cycle\_graph\_initr}
\item
  \texttt{stree}=\texttt{gmgf::stree\_graph\_initr}
\item
  候補辺ソート
\item
  \texttt{basic}=ソートなし
\item
  \texttt{sorted}=\texttt{gmgf::sorted\_graph\_initr}でソート
\item
  状態初期化子
\item
  \texttt{basic}=\texttt{gmgf::basic\_state\_initr}
\item
  \texttt{minmax}=\texttt{gmgf::minmax\_state\_initr}
\item
  \texttt{matrix}=\texttt{gmgf::mmmtr\_state\_initr}
\item
  列挙されたすべてのグラフの頂点間距離の総和が下界に一致したか
\item
  列挙されたグラフの数
\item
  候補辺の数
\item
  展開した状態数
\item
  実行時間 (\texttt{gmgf::fbs\_enumer::enumerate()}の実行時間)
\end{itemize}

\texttt{exp\_cmp\_algo\_enum\_param.py}を用いて,複数のパラメータで実験することが
できる.方法は\texttt{exp\_cmp\_algo.out}の項を参照のこと.

\subsection*{\texorpdfstring{\texttt{exp\_enumer.out}}{exp\_enumer.out}}\label{expux5fenumer.out}

\texttt{exp\_cmp\_algo\_enum.out}で,グラフ初期化子を\texttt{gmgf::stree\_graph\_initr},
状態初期化子を\texttt{gmgf::minmax\_state\_initr}とし,
さらに,列挙された一般化ムーアグラフを出力する機能を付加したプログラム.

\begin{verbatim}
./exp_enumer.out 10 3
\end{verbatim}

の実行例は以下のとおり,

\begin{verbatim}
10,3,2,0,150,1,1,15,22,0.00152
\end{verbatim}

結果は次の値が\texttt{,}で区切られている.

\begin{itemize}
\tightlist
\item
  頂点数
\item
  次数
\item
  \texttt{Q} (\texttt{gmgf::graph\_config::Q()})
\item
  \texttt{R} (\texttt{gmgf::graph\_config::R()})
\item
  頂点間距離の総和の下界
\item
  列挙されたグラフすべての頂点間距離の総和が下界に一致したか
\item
  列挙されたグラフの数
\item
  候補辺の数
\item
  展開した状態数
\item
  実行時間 (\texttt{gmgf::fbs\_enumer::enumerate()}の実行時間)
\end{itemize}

さらに,\texttt{n\textless{}頂点数\textgreater{}-d\textless{}次数\textgreater{}-g\textless{}グラフ番号\textgreater{}-enumer.elist}の形式で,
一般化ムーアグラフの辺リストを保存する.

\subsubsection*{\texorpdfstring{\texttt{exp\_enumer.py}}{exp\_enumer.py}}\label{expux5fenumer.py}

\texttt{exp\_enumer.out}に並列計算とタイムアウト機能を追加したプログラム.
コマンドライン引数は以下のとおり.

\begin{itemize}
\tightlist
\item
  \texttt{-n}=頂点数((次数+1)\ldots{}Nの範囲を対象とする)
\item
  \texttt{-d}=次数(3\ldots{}Dの範囲を対象とする)
\item
  \texttt{-i}=結果ファイルリスト(これらの結果に含まれている頂点数次数組は計算しない)
\item
  \texttt{-o}=結果ファイル(指定しなければ標準出力)
\item
  \texttt{-t}=タイムアウト時間(単位:秒)
\end{itemize}

結果は,\texttt{exp\_enumer.out}の結果を列としたcsvファイルである.
