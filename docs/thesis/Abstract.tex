
\chapter*{概要}
近年の急速なWebアプリケーションの発展とともに,データセンタや大規模グラフ分析が
注目されている.これらの技術はスイッチの相互接続による計算機ネットワークが必要
である.ネットワークの性能はトポロジ,つまり構造によって変化する.
トポロジについて議論する際,ネットワークをグラフとして表現することが多い.
低遅延なネットワークを実現するグラフは,平均頂点間距離の値が小さいグラフで,
そのために平均頂点間距離が小さなグラフの構築が重要である.
平均頂点間距離が理論的な下界と一致するグラフは一般化ムーアグラフと呼ばれる.
現在まで,そのようなグラフを構築する方法がいくつか提案されてきたが,
すべての頂点数と次数の組に対応することと,一般化ムーアグラフの性質を利用して
効率的に構築することの両方を満たす構築方法はまだ提案されていない.
そのため,本研究は,任意の頂点数と次数の組に対応し,グラフの性質を利用して
効率的に一般化ムーアグラフを構築する方法を提案し,その性能を実験にて検証する
ことを目的とする.方法を詳細に説明すると,全探索を基本として,初期状態の変更や
枝刈りによって探索を高効率,高速に行うものである.
実験の結果,頂点数と次数が比較的大きな値(具体的にはそれぞれ$16$と$4$)の場合,
全域木から探索を開始し,直径の下界を用いた枝刈りが有効であることが確認できた.

\begin{comment}
\begingroup
\renewcommand{\cleardoublepage}{}
\renewcommand{\clearpage}{}
\chapter*{Abstract}
\endgroup
In recent years, 

\end{comment}
