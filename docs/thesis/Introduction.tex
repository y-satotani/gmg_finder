
\chapter{序論}
\begin{itemize}
\item 近年,Webアプリケーションの急速な普及に伴い,データセンタが注目されている.
\item データセンタでは,コンピュータクラスタと呼ばれるもので実現されることが多い.
\item また,大規模グラフ分析も注目されている.
\item そのような応用例では,通信タイミングと通信先の計算ノードが予測困難である.
\item この予測不能な通信パターンに対しては,ホップ数が小さいトポロジが有効である.
\item ホップ数が理論的な下界と等しいトポロジを,一般化ムーアグラフと呼ぶ.
\item 一般化ムーアグラフを求める方法はいくつか知られているが,・・・ない.
\item さらに,ある頂点数と次数の正則グラフに一般化ムーアグラフが存在するかを
  求める効率的な方法はまだ知られていない.
\item そこで,本稿では,与えられた頂点数と次数をもつ一般化ムーアグラフが
  存在するかを判定し,存在するなら一般化ムーアグラフを求める方法を提案し,
  検証する.
\end{itemize}
本論文では,グラフをグラフする.
