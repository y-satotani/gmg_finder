
\chapter{序論}

近年,Webアプリケーション,特に,ソーシャルネットワークサービス(SNS)の
発展が進んでいる.それにより,ユーザの活動によって生み出される膨大な量の
データの収集,活用が大いに注目されている.データの収集の技術の一つに,
データセンターが挙げられる.データセンターにはサーバが多数存在し,それぞれの
サーバがスイッチと接続し,さらにスイッチが相互に接続してネットワークを形成している
\cite{Greenberg2009,Al-Fares2008}.
ネットワークの性能はトポロジ,すなわち,スイッチの接続構造によって
変化する.最近,注目されるトポロジに,ランダムグラフを用いたトポロジがある
\cite{Singla2011,Koibuchi2012}.このトポロジは,Fat-Tree\cite{Al-Fares2008}
などの従来のトポロジと比べて,スケーラビリティで有利で,より多くのサーバを
高い稼働率で稼働させることができ,遅延が小さいなどの特徴がある\cite{Singla2011}.
特に,低遅延性はデータセンタのスループットに直結する重要な特性である.

データの活用については,ソーシャルネットワーク分析(SNA)が挙げられる
\cite{Aggarwal2011}.これは,SNSのユーザやWebページを頂点,それらの間
の関連を辺として表したグラフを分析するものである.SNAにはコミュニティ検出や
影響分析などのタスクを含み,それらはマーケティングに応用される.
SNAでは,分析対象となるグラフすべてを一台の計算機に格納するのが困難である
\cite{Ching2015}.そのため,グラフを計算機のネットワークに分散させて
分析を行う\cite{Ching2015,Malewicz2010}.ネットワークは先述のデータセンター
と同じく,計算機とスイッチから構成される.SNAでは,分析中,他の計算機との通信
でグラフのデータの一部を送受信する.このとき,通信のタイミングや通信先の
計算機を前もって予測できない.この状況で,2012年にKoibuchiらは,ホップ数,
つまり,経由するスイッチの数が小さいトポロジが有効であることを示した
\cite{Koibuchi2012}.

これらの例から,ホップ数が小さいネットワークの構築が課題であることが分かる.
ネットワークを,スイッチを頂点とみなしたグラフで表現したとき,ホップ数は
平均頂点間距離と呼ばれる指標となる.すなわち,平均頂点間距離が小さいグラフ
の構築が,遅延を最小限にするネットワークの構築に関連する.現在まで,
平均頂点間距離が小さいグラフの構築に関する研究が行われてきた.
2015年,藤田らは,正則グラフを構築した後,辺の入れ替えを繰り返しながら
グラフの平均頂点間距離を小さくする方法を開発した\cite{Fujita2015}.
ところが,この方法では局所解に陥ることが多く,平均頂点間距離が最小である
グラフを発見することが困難である.

1973年,Cerfらは,平均頂点間距離が小さい正則グラフとして一般化ムーアグラフを
提案し\cite{Cerf1973},翌年に一般化ムーアグラフの平均頂点間距離が下界で
あることを導出した\cite{Cerf1974Lower}.一般化ムーアグラフを求める方法は
いくつか知られている.
2004年,Sampelsは頂点推移グラフの中に一般化ムーアグラフが存在することを応用した
方法を開発した\cite{Sampels2004}.しかし,この方法では,任意の頂点数と次数の
組の一般化ムーアグラフを求めることはできない.
2016年,山本らは,佐藤らが開発した正則グラフを列挙する
方法\cite{Sato2008}を応用し,途中で平均頂点間距離が理論的な下界と一致する
グラフが発見されるまで列挙を繰り返す方法を開発した\cite{Yamamoto2016}.
しかし,この方法は,一般化ムーアグラフがもつ性質を利用して効率的に探索できて
いないと言える.

次数$k$と直径(最大頂点間距離)$\Delta$に対して,頂点数が理論的な上界と一致する
正則グラフはムーアグラフと呼ばれる.ムーアグラフは,頂点数が特別な場合の
一般化ムーアグラフである.これまで,ムーアグラフと一般化ムーアグラフの存在に
関する研究が行われてきた\cite{Miller2005}.1960年,HoffmanとSingletonは,
$\Delta=2$のムーアグラフは$k=2,3,7,57$のときのみ存在することと,
$\Delta=3$のムーアグラフは$k=2$のもののみ存在することを示した
\cite{Hoffman1960}.また,1969年,Friedmanは,$k=3,4,5,6,8$で
$3<\Delta\leq300$の一般化ムーアグラフは存在しないことと,
$3\leq k$で$2\Delta+1$が素数のムーアグラフは存在しないことを示した
\cite{Friedman1971}.McKayらは1979年に,頂点数48,次数3の一般化ムーアグラフ
が存在しないことを\cite{McKay1979}示した.Stantonらは1980年,頂点数44,
次数3の一般化ムーアグラフが存在しないことを示した\cite{Stanton1980}.
このように,すべての頂点数と次数の組に対して,必ず一般化ムーアグラフが
存在するとは限らない.
グラフ理論において,与えられた頂点数と次数に対する一般化ムーアグラフの存在は
未解決で,存在を判定する効率的なアルゴリズムも与えられていない.

そこで,本稿は,与えられた頂点数と次数の一般化ムーアグラフの存在判定と
構築の効率的な方法の提案と,その方法の有効性の検証を目的とする.
まず,一般化ムーアグラフについて,その性質を説明した後,
深さ優先探索をベースにした探索アルゴリズムを示す.
その後,効率化および高速化の手法を提案し,その効果を検証する.
具体的には,探索の初期状態の変更と,枝刈りの導入である.
最後に,初期状態の変更の妥当性を実験により検証する.
