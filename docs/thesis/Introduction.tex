
\chapter{序論}

近年,Webアプリケーションの急速な普及に伴い,データセンタが注目されている.
データセンタでは,それぞれの計算機がスイッチに接続され,スイッチが相互に
接続される.

データセンタの他に,大規模グラフ分析も注目されている.大規模グラフ分析とは,
巨大な道路ネットワークや友達ネットワークをグラフと見なし,その頂点間距離や
中心性などを求めることである.それぞれ地図アプリケーションやマーケティング
に応用される.大規模グラフ分析では,分析対象のネットワークのデータすべてを
一台の計算機に格納するのが困難なため,複数台の計算機による分散解法が
採用されることが多い.

データセンタや大規模グラフ分析では,通信容量の観点から,一台のスイッチや
計算機が接続する機器の数に上限が設けられる.
また,計算機間の通信タイミングと通信先の予測が
困難である.このように予測不能な通信パターンに対しては,
スイッチや計算機から構成される正則グラフとみなしたとき,その平均頂点間距離
が小さいトポロジが有効である\cite{Koibuchi2012,Singla2011}.

平均頂点間距離が理論的な下界と等しいグラフは,一般化ムーアグラフと呼ばれる
\cite{cerf1973computer,Cerf1974}.一般化ムーアグラフを求める方法は
いくつか知られている.ひとつに,正則グラフを構築し,辺の入れ替えを
繰り返しながら平均頂点間距離を小さくする方法\cite{Fujita2015}がある.
また,正則グラフを列挙する方法\cite{Sato2008}を応用し,途中で
理論的な下界と一致するものが発見されるまで列挙を繰り返す方法がある.
さらに,Vertex transiveなグラフの中に一般化ムーアグラフがあることを
応用した方法もある\cite{Sampels2004}.

しかし,いずれの方法も一般化ムーアグラフの性質を考慮し,効率的に探索が
できていない.そこで,本稿では,与えられた頂点数と次数の一般化ムーアグラフが
存在するかを判定し,存在するなら一般化ムーアグラフを求める方法を提案し,
検証する.
