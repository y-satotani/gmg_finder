
\chapter{実験プログラム}
\label{chap:experimental-program}
実験に用いたプログラムとその実行方法について説明する.
プログラムは,\verb|github.com/y-satotani/gmg-finder|にある.
まず,実験プログラムのビルド手順を示した後,それぞれの実験プログラムについて
説明する.

まず,実験プログラムのビルド手順を示す.
\begin{enumerate}
\item 必要なソフトウェアやライブラリをインストールする
  \begin{itemize}
  \item \verb|git|
  \item \verb|gcc|
  \item \verb|igraph|
  \end{itemize}
\item \verb|git|でクローンする
  (\verb|git clone github.com/y-satotani/gmg-finder.git|)
\item \verb|src|ディレクトリに移動(\verb|cd gmg-finder/src|)
\item \verb|gmgf/Makefile|と\verb|test/Makefile|と
  \verb|experiment/Makefile|の編集
  (\verb|IDIR|と\verb|LDIR|の値を\verb|igraph|がある場所に設定する)
\item \verb|make|の実行(\verb|make ; make install|)
\end{enumerate}
ビルドに成功すると,実験プログラムとテストプログラムが\verb|src/bin|ディレクトリ
に現れる.

以降,これらの実験プログラムについて説明する.
\begin{enumerate}
\item \texttt{exp\_cmp\_algo.out}
\item \texttt{exp\_cmp\_algo\_full.out}
\item \texttt{exp\_dist\_delete.out}
\item \texttt{exp\_miner.out}
\end{enumerate}

\section*{\texttt{exp\_cmp\_algo.out}}
\texttt{exp\_cmp\_algo.out}は,初期グラフと枝刈りの方法を変えて性能を測定する
プログラムである.実行例と結果例を以下に示す.
\begin{verbatim}
experiment/exp_cmp_algo.out basic basic basic 10 3
10,3,2,0,150,basic,basic,basic,150,15,16,0.000144
\end{verbatim}
結果は次の値が'',''で区切られる.
\begin{enumerate}
\item 頂点数
\item 次数
\item $Q$(式\ref{eq:gmg-q})
\item $R$(式\ref{eq:gmg-r})
\item 頂点間距離の総和の下界(定理\ref{thm:gmg-lower-bound})
\item 初期グラフ
  \begin{itemize}
  \item basic$\cdots$基本初期グラフ(定義\ref{def:basic-initial-graph})
  \item cycle$\cdots$閉路初期グラフ(定義\ref{def:cycle-initial-graph})
  \item stree$\cdots$全域木初期グラフ(定義\ref{def:stree-initial-graph})
  \end{itemize}
\item 候補辺ソート
  \begin{itemize}
  \item basic$\cdots$ソートなし
  \item sorted$\cdots$アルゴリズム\ref{algo:sort-edges}でソート
  \end{itemize}
\item 枝刈り
  \begin{itemize}
  \item basic$\cdots$枝刈りなし
  \item minmax$\cdots$最大グラフを利用した枝刈り
    (\ref{sect:diameter-lower-bound}節)
  \item matrix$\cdots$頂点間距離の更新を利用した枝刈り
    (\ref{sect:fast-distance-update}節)
  \end{itemize}
\item 結果のグラフの頂点間距離の総和
\item 候補辺の数
\item 展開状態数
\item 探索時間(単位:秒)
\end{enumerate}

次のコマンドで,複数の結果を含むcsvファイルを生成する.
\verb|--max-procs=x|の\verb|x|の値だけ並列に実行する.
\begin{verbatim}
(echo n,d,Q,R,sspl_lb,ginitr,sorted,sinitr,sspl,edge,node,time ;\
 experiment/exp_cmp_algo_param.py |\
 xargs --max-lines=1 --max-procs=4 experiment/exp_cmp_algo.out) >\
cmp-algo.csv
\end{verbatim}
パラメータは,\texttt{exp\_cmp\_algo\_param.py}で生成される.
そのため,別のパラメータで実験する場合は,このファイルを編集すればよい.

\section*{exp\_cmp\_algo\_full.out}
\texttt{exp\_cmp\_algo.out}では探索開始から(もしあれば)最初の一個の
一般化ムーアグラフを得るまでの情報を計測したが,
\texttt{exp\_cmp\_algo\_full.out}では探索開始からすべての一般化ムーアグラフ
を得るまでの情報を計測する.実行例と実行結果例は次のとおり.

\begin{verbatim}
./exp_cmp_algo_full.out basic basic basic 10 3
10,3,2,0,150,basic,basic,basic,4,150,150,15,75,0.00042
\end{verbatim}

結果は,次の値が'',''で区切られている.
\begin{enumerate}
\item 頂点数
\item 次数
\item $Q$
\item $R$
\item 頂点間距離の総和の下界
\item 初期グラフ (\texttt{exp\_cmp\_algo.out}と同じ)
\item 候補辺ソート (\texttt{exp\_cmp\_algo.out}と同じ)
\item 枝刈り (\texttt{exp\_cmp\_algo.out}と同じ)
\item 結果のグラフの頂点間距離の総和
\item 得られたグラフの数 (同型グラフを重複して数える)
\item 頂点間距離の総和の最大値
\item 頂点間距離の総和の最小値
\item 候補辺の数
\item 展開状態数
\item 探索時間(単位:秒)
\end{enumerate}

\section*{exp\_dist\_delete.out}
\texttt{exp\_dist\_delete.out}は一辺削除時の頂点間距離の更新の実験を行う.
次の項目を入力とする.
\begin{enumerate}
\item ネットワークトポロジ
  \begin{itemize}
  \item \verb|random|=Erdos-R{\'e}nyiモデル
  \item \verb|scale-free|=Barab{\'a}si-Albertモデル
  \end{itemize}
\item 頂点数
\item ネットワークパラメータ
\item シード値
\end{enumerate}

コマンドの例と実行結果の例は以下のとおり.
\begin{verbatim}
./exp_dist_delete.out random 1000 0.3 0
1000,0.3,random,0,150422,1,0.827778,0.076536
\end{verbatim}

結果は次の値が`,`で区切られている.
\begin{enumerate}
\item 頂点数
\item ネットワークパラメータ (トポロジにより,$p$か$m$)
\item ネットワークトポロジ
  \begin{itemize}
  \item \verb|random|=Erdos-R{\'e}nyiモデル
  \item \verb|scale-free|=Barab{\'a}si-Albertモデル
  \end{itemize}
\item シード値
\item 辺の数
\item 結果が一致したかどうか (1なら一致)
\item \verb|igraph|ライブラリでの計算時間
\item 提案手法の計算時間
\end{enumerate}

\section*{exp\_miner.out}
\texttt{exp\_cmp\_algo.out}を,初期グラフを全域木に,枝刈りを
\ref{sect:fast-distance-update}節で示した方法に固定したプログラムである.
さらに,(もしあれば)一般化ムーアグラフを出力する機能を付加した.

実行結果と実行例は次のとおり.
\begin{verbatim}
./exp_miner.out 12 3
12,3,2,2,252,matrix,252,26,27,0.000925
\end{verbatim}

標準出力の結果は次の値が'',''で区切られている.
\begin{enumerate}
\item 頂点数
\item 次数
\item $Q$
\item $R$
\item 頂点間距離の総和の下界
\item 使用アルゴリズム(\verb|matrix|で固定)
\item 結果のグラフの頂点間距離の総和
\item 候補辺の数
\item 展開した状態数
\item 探索時間(単位:秒)
\end{enumerate}

さらに,''\texttt{n<頂点数>-d<次数>-miner.elist}''のファイル名で,
一般化ムーアグラフの辺リストを保存する.出力例は以下のとおり.
\begin{multicols}{4}
  \verbatiminput{n12-d3-example.elist}
\end{multicols}
ただし,辺のリストを4列で表しているが実際は1列である.

\subsection*{\texttt{exp\_miner.py}}
\texttt{exp\_miner.out}に並列計算とタイムアウト機能を追加したプログラムである.
コマンドライン引数は以下のとおり.
\begin{itemize}
\item \texttt{-n}$\cdots$頂点数((次数+1)...Nの範囲を対象とする)
\item \texttt{-d}$\cdots$次数(3...Dの範囲を対象とする)
\item \texttt{-i}$\cdots$結果ファイルリスト(これらの結果に含まれている頂点数次数組は計算しない)
\item \texttt{-o}$\cdots$結果ファイル(指定しなければ標準出力)
\item \texttt{-t}$\cdots$タイムアウト時間(単位:秒)
\end{itemize}
結果は,\texttt{exp\_miner.out}の結果を列としたcsvファイルである.

\chapter{一般化ムーアグラフのリスト}
\label{chap:list-of-gmg}
付録\ref{chap:experimental-program}で説明した\verb|exp-miner.out|を
用いて,与えられた頂点数と次数の組の一般化ムーアグラフが存在するかを
求めた.その結果を図\ref{fig:gmg-existence}に示す.
ただし,予想\ref{conj:spanning-tree}の結果次第で,存在しない組が無効になる
可能性があることに注意する.ちなみに,一般化ムーアグラフの実体は
\begin{itemize}
\item \verb|github.com/y-satotani/gmg-finder/tree/master/docs/res/graph|
\end{itemize}
で公開している.

\begin{figure}[htbp]
  \centering
  \captionsetup{justification=centering}
  \includegraphics{gmg-existence.pdf}
  \caption{頂点数と次数の一般化ムーアグラフの存在\\
    空白部分は,正則グラフが存在しない組合せもしくは,
    まだ終了していない組合せを表す.}
  \label{fig:gmg-existence}
\end{figure}
