
\renewcommand{\thesubsection}{(\alph{subsection})}

\graphicspath{{../res/figure/}{../res/plot-the/}}
\makeatletter
\def\input@path{{../res/figure/}{../res/graph/}{../res/table-the/}}
\makeatother

% setting margin
\newgeometry{tmargin=2cm,lmargin=2cm,rmargin=2cm,bmargin=2cm}

% adding new symbol (vrectangle and vrectangleblack)
\DeclareFontEncoding{LS1}{}{}
\DeclareFontSubstitution{LS1}{stix}{m}{n}
\DeclareSymbolFont{symbolsstix}{LS1}{stixscr}{m}{n}
\DeclareMathSymbol{\vrectangleblack}{\mathord}{symbolsstix}{"C5}
\DeclareMathSymbol{\vrectangle}{\mathord}{symbolsstix}{"C6}

% add math operator
\DeclareMathOperator*{\argmin}{\arg\!\min}
\DeclareMathOperator*{\argmax}{\arg\!\max}

% theorem preference
\declaretheoremstyle[
  spaceabove=1ex, 
  numbered=yes,
  headfont=\bfseries,
  headpunct=,
  bodyfont=\normalfont,
  spacebelow=1ex,
]{mythmstyle}
\declaretheorem[
  parent=chapter,style=mythmstyle,qed=$\vrectangle$,title=定義,
]{definition}
\declaretheorem[
  style=mythmstyle,sibling=definition,qed=$\vrectangle$,title=例,
]{example}
\declaretheorem[
  style=mythmstyle,sibling=definition,title=補題,
]{lemma}
\declaretheorem[
  style=mythmstyle,sibling=definition,qed=$\vrectangle$,title=補題,
]{lemma-without-proof}
\declaretheorem[
  style=mythmstyle,sibling=definition,title=定理,
]{theorem}
\declaretheorem[
  style=mythmstyle,sibling=definition,qed=$\vrectangle$,title=定理,
]{theorem-without-proof}
\declaretheorem[
  style=mythmstyle,sibling=definition,title=系,
]{collary}
\declaretheorem[
  style=mythmstyle,sibling=definition,qed=$\vrectangle$,title=系,
]{collary-without-proof}
\declaretheorem[
  style=mythmstyle,sibling=definition,qed=$\vrectangle$,title=予想,
]{conjecture}
\renewcommand{\proofname}{\normalfont{[\,証明\,]}\nopunct}
\renewcommand{\qedsymbol}{$\vrectangleblack$}

% algorithm preference
\renewcommand{\listalgorithmname}{アルゴリズムリスト}
\renewcommand{\thealgorithm}{\arabic{chapter}.\arabic{algorithm}}
\makeatletter
\renewcommand{\ALG@name}{アルゴリズム}
\@addtoreset{algorithm}{chapter}
\makeatother

% minted preference
\usemintedstyle{friendly}

% bibtex preference
\renewcommand{\bibname}{参考文献}
\bibliographystyle{junsrt}
