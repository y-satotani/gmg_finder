
\chapter{結論}
本稿では,全探索法を元にした一般化ムーアグラフの探索の高速化および高効率化に
ついて述べた.まず,一般化ムーアグラフの性質を示して証明を与えた.
次に,この性質を利用して,一般化ムーアグラフを探索するアルゴリズムを定めた.
最後に,高速化および高効率化の方法を与え,それに対する結果を示した.

本論文では,高速化および高効率化の方法として,探索の初期状態を変更することと,
枝刈りによって探索空間を削減することの二つを提案した.
探索の初期状態の変更とは,初期状態のグラフに予めいくつかの辺追加した状態を
新たな初期状態とすることである.本稿では,辺を二本追加して閉路を含むようにした
グラフと,全域木の二種類を提案した.実験の結果から,一般的な場合では全域木から
探索を始める場合が最も効率が良いことが分かった.ただし,$R=2$の場合では閉路を
含むグラフから始める場合が効率が良いことが示された.

枝刈りの方法として,探索途中のグラフの直径の下界を用いる方法を提案した.
また,この方法の派生として,辺の追加と削除に対する頂点間距離の
高速な更新法を応用した方法も提案した.探索空間の削減によって高効率化は
達成されたが,頂点数および次数が小さい場合に高速化は達成されないことが
結果から示された.

今後の目標は,提案した方法を用いて一般化ムーアグラフの存在を調べることや,
提案した初期状態から探索を開始することの妥当性の証明である.
