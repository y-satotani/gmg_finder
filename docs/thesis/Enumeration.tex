
\chapter{一般化ムーアグラフの列挙}
本章では,第\ref{chap:reduce-by-initial-graph}章で導入した初期グラフの
妥当性(予想\ref{conj:gmg-cycle}と予想\ref{conj:spanning-tree})を検証する.
そのために,一般化ムーアグラフを重複なく列挙する方法を与え,実験を行う.

\section{列挙アルゴリズム}
\label{sect:enum-algorithm}
一般化ムーアグラフを重複なく列挙する,つまり,非同型なグラフを列挙する
方法を与える.そのためのアルゴリズムをアルゴリズム\ref{algo:gmg-enumeration}
に示す.ここで,$\text{Aut}(G)$をグラフ$G$の同型グラフの集合とする.

\begin{algorithm}[H]
  \caption{一般化ムーアグラフの列挙アルゴリズム}
  \label{algo:gmg-enumeration}
  \begin{algorithmic}[1]
    \Require $n,k$
    \Ensure 互いに非同型な$M(n,k)$の集合
    \Procedure{EnumerateGeneralizedMooreGraph}{}
    \State $G_I\gets\text{初期グラフ}$
    \State $\{e_i\}_{i\in\mathbb{N}}^M\gets G_I\text{の候補辺}$
    \State $Gs_1\gets\{G_I\}$
    \ForAll{$i\in\{1,\ldots,M\}$}
    \State $Gs_{i+1}\gets\varnothing$
    \ForAll{$G\in Gs_i$}
    \ForAll{$operator\in\{\text{追加オペレータ},\text{無追加オペレータ}\}$}
    \If{$G,e_i$に$operator$が適応でき,かつ,
      $operator(G,e_i)\notin\cup_{H\in Gs_{i+1}}\text{Aut}(H)$}
    \State $Gs_{i+1}\gets Gs_{i+1}\cup\{operator(G,e_i)\}$
    \EndIf
    \EndFor
    \EndFor
    \EndFor
    \State \textbf{return} $\{G\,|\,G\in Gs_{M+1},$
    $G\text{が正則で定理\ref{thm:gmg-geometric-property}を満たす}\}$
    \EndProcedure
  \end{algorithmic}
\end{algorithm}

\section{実験}
節\ref{sect:enum-algorithm}で定義したアルゴリズムを用いて,一般化ムーアグラフを
列挙し,第\ref{chap:reduce-by-initial-graph}章で導入した初期グラフの妥当性を
検証する.具体的には,初期グラフを変更してアルゴリズムを開始し,得られた
一般化ムーアグラフの個数が同じかを確認する.
検証する頂点数$n$と次数$k$は,$k=3,4$と
\begin{equation*}
  \begin{aligned}
    n=\begin{cases}
      4,6,8,10,12,14,16,18 & (k=3) \\
      5,6,7,8,9,10,11 & (k=4)
    \end{cases}
  \end{aligned}
\end{equation*}
とする.実験を行った環境は,表\ref{tab:env-lab}の通りである.

\section{結果}
アルゴリズムにより得られた一般化ムーアグラフの数を表\ref{tab:ginitr-ngraph-iso}
に示す.

% latex table generated in R 3.4.2 by xtable 1.8-2 package
% Thu Jan  4 15:06:44 2018
\begin{table}[ht]
\centering
\caption{初期グラフを変更したときの互いに非同型な一般化ムーアグラフの数} 
\label{tab:ginitr-ngraph-iso}
\begin{tabular}{lrrr}
  \hline
$(n,d)$ & 基本初期グラフ & 閉路初期グラフ & 全域木初期グラフ \\ 
  \hline
(4,3) &   1 &   1 &   1 \\ 
  (6,3) &   2 &   2 &   2 \\ 
  (8,3) &   2 &   2 &   2 \\ 
  (10,3) &   1 &   1 &   1 \\ 
  (12,3) &   2 &   2 &   2 \\ 
  (14,3) &   7 &   7 &   7 \\ 
  (16,3) &   6 &   6 &   6 \\ 
  (18,3) &   1 &   1 &   1 \\ 
  (5,4) &   1 &   1 &   1 \\ 
  (6,4) &   1 &   1 &   1 \\ 
  (7,4) &   2 &   2 &   2 \\ 
  (8,4) &   6 &   6 &   6 \\ 
  (9,4) &  16 &  16 &  16 \\ 
  (10,4) &  24 &  24 &  24 \\ 
  (11,4) &  37 &  37 &  37 \\ 
   \hline
\end{tabular}
\end{table}


結果のとおり,初期グラフを変更したことによって列挙されない一般化ムーアグラフは
存在しないことが分かる.つまり,実験を行った頂点数と次数の範囲では,
初期グラフの変更は妥当である.
