
\chapter{一般化ムーアグラフのリスト}
\label{chap:list-of-gmg}
付録\ref{chap-experimental-program}で説明した\verb|exp-miner.out|を
用いて,与えられた頂点数と次数の組の一般化ムーアグラフが存在するかを
求めた.その結果を図\ref{fig:gmg-existence}に示す.
ただし,予想\ref{conj:spanning-tree}の結果次第で,存在しない組が無効になる
可能性があることに注意する.ちなみに,一般化ムーアグラフの実体は,次の
URLにて公開している.

\url{github.com/y-satotani/gmg-finder/tree/master/docs/res/graph}

\begin{figure}[htbp]
  \centering
  \captionsetup{justification=centering}
  \includegraphics{gmg-existence-v.pdf}
  \caption{頂点数と次数の一般化ムーアグラフの存在\\
    空白部分は,正則グラフが存在しない組合せもしくは,
    まだ終了していない組合せを表す.}
  \label{fig:gmg-existence}
\end{figure}
