\usepackage{amsmath,amsthm,amssymb}
\usepackage{graphicx}
\usepackage{subfig}
\usepackage{bxdpx-beamer}
\usepackage{pxjahyper}
\usepackage{appendixnumberbeamer}

\usetheme{Madrid}
%\usetheme{Rochester}
\setbeamertemplate{itemize items}[square]
\setbeamertemplate{enumerate items}[square]
\setbeamertemplate{bibliogrpahy item}[text]
\renewcommand{\kanjifamilydefault}{\gtdefault}
\renewcommand{\familydefault}{\sfdefault}
\usefonttheme[onlymath]{serif}
%\setbeamertemplate{navigation symbols}{}
%\setbeamertemplate{footline}[frame number]
\setbeamertemplate{caption}[numbered]
\setbeamertemplate{caption label separator}[space]
\setbeamertemplate{bibliography item}[text]
\setbeamertemplate{theorems}[numbered]
\renewcommand{\figurename}{図}
\renewcommand{\tablename}{表}
\renewcommand{\appendixname}{補足}
\theoremstyle{definition}
\newtheorem{thm}{定理}
\newtheorem{conj}[thm]{予想}
\graphicspath{{../res/figure/}{../res/table/}}
\makeatletter
\def\input@path{{../res/figure/}{../res/table/}}
\makeatother
\bibliographystyle{jalpha}

\title{一般化ムーアグラフの探索の高効率化}
\author{里谷 佳紀}
\institute{高橋研究室}
\date[特別研究中間発表会]{特別研究中間発表会 \\ 2017年11月15日}
