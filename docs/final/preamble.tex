\usepackage{amsmath,amsthm,amssymb}
\usepackage{thmtools}
\usepackage{graphicx}
\usepackage{subcaption}
\usepackage{bxdpx-beamer}
\usepackage{pxjahyper}
\usepackage{appendixnumberbeamer}
\usepackage{algorithm,algpseudocode}

%\usetheme{Madrid}
\usetheme{Boadilla}
\usecolortheme{rose}
\setbeamertemplate{itemize items}[square]
\setbeamertemplate{enumerate items}[square]
\setbeamertemplate{bibliogrpahy item}[text]
\renewcommand{\kanjifamilydefault}{\gtdefault}
\renewcommand{\familydefault}{\sfdefault}
\usefonttheme[onlymath]{serif}
\setbeamertemplate{navigation symbols}{}
%\setbeamertemplate{footline}[frame number]
\setbeamertemplate{caption}[numbered]
\setbeamertemplate{caption label separator}[space]
\setbeamertemplate{bibliography item}[text]
\setbeamertemplate{theorems}[numbered]
\setbeamertemplate{frametitle continuation}[from second][(続き)]
\setbeamertemplate{blocks}[default]
\setbeamercolor{block title}{bg=gray!20}
\setbeamercolor{block body}{bg=gray!10}

\renewcommand{\figurename}{図}
\renewcommand{\tablename}{表}
\renewcommand{\appendixname}{補足}
\let\definition\relax
\let\theorem\relax
\let\conjecture\relax
\declaretheorem[title=定義]{definition}
\declaretheorem[sibling=definition,title=定理]{theorem}
\declaretheorem[sibling=definition,title=予想]{conjecture}

\graphicspath{{../res/figure/}{../res/plot-fin/}{../res/figure-fin/}}
\makeatletter
\def\input@path{{../res/figure/}{../res/plot-fin/}{../res/figure-fin/}}
\makeatother
\bibliographystyle{jalpha}

\title{一般化ムーアグラフの探索の高効率化}
\author{里谷 佳紀}
\institute{高橋研究室}
\date[特別研究発表会]{特別研究発表会 \\ 2018年2月16日}
