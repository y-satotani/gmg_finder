\begin{table}[H]
  \centering
  \caption{実験結果 \\ 横に並んだ数字は頂点数を,縦に並んだ数字は次数を表す.
    \checkmark はプログラムが最初に発見したグラフがCerfらの下界を達成する
    組み合わせを,NAは正則グラフが存在しない組み合わせを表す.}
  \begin{tabular}{|c|c|c|c|c|c|c|c|c|c|}
    \hline
    & 4 & 5 & 6 & 7 & 8 & 9 & 10 & 11 & 12 \\ \hline
    3 & \checkmark & NA & \checkmark & NA & \checkmark & NA & \checkmark & NA & \checkmark \\ \hline
    4 & NA & \checkmark & \checkmark & \checkmark & \checkmark & \checkmark & \checkmark & \checkmark & \checkmark \\ \hline
    5 & NA & NA & \checkmark & NA & \checkmark & NA & \checkmark & NA & \checkmark \\ \hline
  \end{tabular}
  \label{tab:result}
\end{table}
