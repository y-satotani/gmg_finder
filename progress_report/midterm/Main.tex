\documentclass[dvipdfmx]{beamer}
\usepackage{amsmath,amsthm,amssymb}
\usepackage{graphicx}
\usepackage{subfig}
\usepackage{bxdpx-beamer}
\usepackage{pxjahyper}
\usepackage{appendixnumberbeamer}

\usetheme{Madrid}
%\usetheme{Rochester}
\setbeamertemplate{itemize items}[square]
\setbeamertemplate{enumerate items}[square]
\renewcommand{\kanjifamilydefault}{\gtdefault}
\renewcommand{\familydefault}{\sfdefault}
%\setbeamertemplate{navigation symbols}{}
%\setbeamertemplate{footline}[frame number]
\setbeamertemplate{caption}[numbered]
\setbeamertemplate{caption label separator}[space]
\setbeamertemplate{bibliography item}[text]
\setbeamertemplate{theorems}[numbered]
\renewcommand{\figurename}{図}
\renewcommand{\tablename}{表}
\renewcommand{\appendixname}{補足}
\theoremstyle{definition}
%\newtheorem{theorem}{定理}
\newtheorem{conjecture}[theorem]{予想}
\graphicspath{{../../thesis/figure/}}
\makeatletter
\def\input@path{{../../thesis/figure/}}
\makeatother

\title{一般化ムーアグラフの探索の高効率化}
\author{里谷 佳紀}
\institute{高橋研究室}
\date[特別研究中間発表会]{特別研究中間発表会 \\ 2017年11月15日}

\begin{document}
\begin{frame}
  \maketitle
\end{frame}

\section{本編}
\begin{frame}{背景と目的}
  \begin{itemize}
  \item データセンタではクラスタが用いられる
  \item アプリケーションの通信のタイミングや通信先は予測は困難
  \item 単純にホップ数が小さいトポロジが有効\cite{Koibuchi2012, Singla2011}
  \item \alert{一般化ムーアグラフ}$\cdots$
    ホップ数と平均頂点間距離が理論的に下界な正則グラフ
    \cite{cerf1973computer, Cerf1974}
  \item 一般化ムーアグラフを発見するアルゴリズムはいくつかある
    \cite{Sampels2004, 2015, 2016}
  \item 次をすべて満たす効率的な発見方法はまだ知られていない
    \begin{itemize}
    \item (可能な)頂点数と次数をすべて網羅
    \item ランダム性がない
    \end{itemize}
  \item 全探索をベースにして効率的に一般化ムーアグラフを
    探索する方法を提案・検証することが目的
  \end{itemize}
\end{frame}

\begin{frame}{方法}
  \begin{enumerate}
  \item 与えられた頂点数と次数の初期グラフを構築しておく
  \item 図\ref{fig:feasible-edges-example}の破線部分の辺を
    追加できる(できない)か求め真の場合次のグラフを生成
    \begin{itemize}
    \item 図\ref{fig:feasible-edges-example2}では最小の閉路長が
      $5$以上のとき追加できる
      \begin{enumerate}[a]
      \item 3-閉路ができるので追加できない
      \item 4-閉路ができるので追加できない
      \item 最小の閉路長は6なので追加できる
      \end{enumerate}
    \end{itemize}
    \item 初期グラフを変えることで探索を高効率化
  \end{enumerate}
  \begin{figure}
    \centering
    \begin{minipage}{.4\columnwidth}
      \def\svgwidth{\textwidth}
      \input{feasible-edges-example.pdf_tex}
      \captionof{figure}{初期グラフの例と辺の候補}
      \label{fig:feasible-edges-example}
    \end{minipage}
    \hspace{1em}
    \begin{minipage}{.4\columnwidth}
      \def\svgwidth{\textwidth}
      \input{feasible-edges-example2.pdf_tex}
      \captionof{figure}{探索途中のグラフ}
      \label{fig:feasible-edges-example2}
    \end{minipage}
  \end{figure}
\end{frame}

\begin{frame}{現在までの成果}
  \begin{enumerate}
  \item いち
  \item に
  \item さん
  \end{enumerate}
\end{frame}

\begin{frame}{今後の課題}
  \begin{enumerate}
  \item 直径の下界を利用した枝刈り
  \item 予想の証明
    \begin{conjecture}
      hogehoge
    \end{conjecture}
    \begin{conjecture}
      hogehoge
    \end{conjecture}
  \item 同型グラフ判定を用いた列挙
  \item 一般化ムーアグラフが存在しない場合の考察
  \end{enumerate}
\end{frame}

\appendix
\begin{frame}{実行時間の比較}
  \begin{itemize}
  \item foo
  \item bar
  \end{itemize}
\end{frame}

\begin{frame}[allowframebreaks]{参考文献}
  \bibliographystyle{jalpha}
  \bibliography{MyCollection}
\end{frame}

\end{document}

