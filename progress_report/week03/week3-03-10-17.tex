\documentclass[11pt]{jarticle}
%
\usepackage[dvipdfmx]{graphicx}
\usepackage{amsmath}
\usepackage{amssymb}
\usepackage{bm}
\usepackage{latexsym}
\usepackage{float}
\usepackage{hyperref}
\usepackage{color}
\usepackage[justification=centering]{caption}
%
\addtolength{\textwidth}{40mm}
\addtolength{\oddsidemargin}{-20mm}
\addtolength{\evensidemargin}{-20mm}
\addtolength{\textheight}{15mm}
\addtolength{\topmargin}{-10mm}
%
\pagestyle{empty}
%
\title{研究進捗報告}
\author{里谷 佳紀}
%\date{\today}
\date{平成29年10月3日}
%
\begin{document}
%
\maketitle
\thispagestyle{empty}
%
\section{研究全体の目標}
与えられた頂点数と次数をもつ正則グラフのうち,Cerfらの平均頂点間距離の
下界\cite{Cerf1974}と一致する平均頂点間距離をもつグラフが存在するかを
判定する方法を開発する.また,既存の方法\cite{Yamamoto2016}と比較する
ことにより,新方法の有用性を検証する.

\section{前回打ち合わせ時に定めた短期目標}
\begin{enumerate}
\item $d=3$,$R=2$の正則グラフにおいて,あらかじめ$Q+1$層の頂点ひとつを
  $Q$層の頂点2個と,$2Q$以下の長さの閉路を持たないように隣接させたグラフから
  探索を開始するプログラムを実装し,前回実装したプログラムとの比較をおこなう.
\end{enumerate}

\section{本日までの進捗状況}
\begin{enumerate}
\item プログラムを実装し,比較を行った.$n=12$のときと,$n=24$のときは,
  両プログラムとも,およそ10msで終了したため,時間での比較はできなかった.
  $n=48$のときは,今回のプログラムはおよそ15時間で終了した.対して,
  前回のプログラムは今回のミーティングまでに終了しなかった($>$170時間). \\
  ちなみに,プログラムの結果から,$n=48$のときに,
  条件(長さ$2Q$以下の閉路が存在せず,直径が$Q+1$)を満たすグラフは
  存在しないことが分かった. \\
\end{enumerate}


%\begin{figure}
%  \centering
%  \resizebox{75mm}{!}{\input{week3-tree1.pdf_tex}}
%  \caption{tree1}
%  \label{fig:tree-naive}
%\end{figure}

%\begin{figure}
%  \centering
%  \resizebox{75mm}{!}{\input{week3-tree2.pdf_tex}}
%  \caption{tree2}
%  \label{fig:tree-r2}
%\end{figure}

\bibliographystyle{junsrt}
\bibliography{refs}

\end{document}
