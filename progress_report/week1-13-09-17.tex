\documentclass[11pt]{jarticle}
%
\usepackage{graphicx}
\usepackage{amsmath}
\usepackage{amssymb}
\usepackage{bm}
\usepackage{latexsym}
%
\addtolength{\textwidth}{40mm}
\addtolength{\oddsidemargin}{-20mm}
\addtolength{\evensidemargin}{-20mm}
\addtolength{\textheight}{15mm}
\addtolength{\topmargin}{-10mm}
%
\pagestyle{empty}
%
\title{研究進捗報告}
\author{里谷 佳紀}
\date{\today}
%\date{平成27年10月21日}
%
\begin{document}
%
\maketitle
\thispagestyle{empty}
%
\section{研究全体の目標}
与えられた頂点数と次数をもつ正則グラフのうち,Cerfらの平均頂点間距離の下界\cite{Cerf1974}と一致する
平均頂点間距離をもつグラフが存在するかを判定する方法を開発する.
また,既存の方法\cite{Yamamoto2016}と比較することにより,新方法の有用性を検証する.

\section{前回打ち合わせ時に定めた短期目標}
\begin{enumerate}
\item 二分決定木,ZDD,フロンティア法の理解
\item 先行研究の結果の確認
\end{enumerate}

\section{本日までの進捗状況}
\begin{enumerate}
\item ZDDとフロンティア法の基本事項をノートにまとめた.
\item 先行研究が検証した,頂点数と次数の組をノートにまとめた.
\item 
\end{enumerate}

\bibliographystyle{jalpha}
\bibliography{refs}

\end{document}
