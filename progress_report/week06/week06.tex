\documentclass[11pt]{jarticle}
%
\usepackage[dvipdfmx]{graphicx}
\usepackage{amsmath}
\usepackage{amssymb}
\usepackage{bm}
\usepackage{latexsym}
\usepackage{float}
\usepackage{subfig}
\usepackage[justification=centering]{caption}
%
\addtolength{\textwidth}{40mm}
\addtolength{\oddsidemargin}{-20mm}
\addtolength{\evensidemargin}{-20mm}
\addtolength{\textheight}{15mm}
\addtolength{\topmargin}{-10mm}
%
\pagestyle{empty}
%
\title{研究進捗報告}
\author{里谷 佳紀}
%\date{\today}
\date{平成29年10月24日}
%
\begin{document}
%
\maketitle
\thispagestyle{empty}
%
\section{研究全体の目標}
深さ優先探索をベースにした,一般化ムーアグラフを発見するアルゴリズムを開発する.
さらに,初期グラフの改良や枝刈りを導入する,更なる改良案を提案する.
同時に,これらの改良を評価する.
\section{前回打ち合わせ時に定めた短期目標}
\begin{enumerate}
\item 展開ノード数計測の実験の続き
\item 辺の削除に伴う最短距離の更新の実験
\item $\sigma'_{st}=\sum_v\sigma'_{sv}\sigma'_{vt}/(d'_{st}-1)$の証明
\item 定理とプログラムの文書化
  \begin{enumerate}
  \item $2Q$以下の閉路が存在しないことの定理
  \item 基本アルゴリズム
  \end{enumerate}
\end{enumerate}
\section{本日までの進捗状況}
\begin{enumerate}
\item 現在の結果を図\ref{fig:result-ext-nodes}に示す.
  未だ計算が終わっていない組み合わせがある.
\item ランダムネットワーク(Erdos-R{\'e}nyiモデル)と
  スケールフリーネットワーク(Barab{\'a}si-Albertモデル)上で
  実験を行った.結果を図\ref{fig:result}に示す.
  密グラフでは性能が上がり,疎グラフでは効果があまりないことが分かった.
\item アイディアを打ち合わせ中に説明する.
\item 完成したものがある.
  \begin{enumerate}
  \item 未着手
  \item 部分的に書いた.
  \end{enumerate}
\end{enumerate}

\begin{figure}[H]
  \centering
  \subfloat[ランダムネットワーク]{
    \includegraphics[width=.7\linewidth]{exp_dist_delete_result_random.pdf}
  }\hfill
  \subfloat[スケールフリーネットワーク]{
    \includegraphics[width=.7\linewidth]{exp_dist_delete_result_scale_free.pdf}
  }
  \caption{ネットワークトポロジ別の実行時間 \\
    ネットワークのパラメータで色分けした.
    パラメータ別に100回実験し平均を求めた.
    実線はigraphライブラリの\texttt{igraph\_shortest\_paths()}の
    実行時間を,破線は提案手法の実行時間を表す.
  }
  \label{fig:result}
\end{figure}

\end{document}
